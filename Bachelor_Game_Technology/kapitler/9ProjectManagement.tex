\chapter{Project Management}
\label{ch:kap9}

\section{Planning}
\label{sec:planning}
The planning is often the basis for how well the project will succeed, or if it would succeed at all. If the work here here is neglected, will the outcome be expensive in terms of time and workflow. The gain of good planning and being well prepared on the other hand, is a motivated team ready to start the project. This quote said by Benjamin Franklin \textit{``If you fail to prepare, prepare to fail"} proves the importance of preparing. By investing some time and effort in the early stages of the preparation, you will end up saving time which will increase the probability for success.
\cite[P.~92]{ProjectManagement}

\subsection{Team Composition}
\label{sec:teamComp}
When the team has already set makes a great opportunity to work with new people. There's a opportunity to learn a lot about yourself and others. The group would also have to consider and handle the different kinds of personalities, and how to solve task and challenges, when you work with a team of new people. There is also an extra challenge to work in a multidisciplinary team, that is because the team members have different specialties and therefore also work methods. There is a lot of things to consider when working in a team such as competence, experience, cooperative skills, preferences and weaknesses. We all have our weaknesses, and it's important to think about how it can effect the team. Motivation, ambitions and work capacity also plays a great factor when it comes to teamwork. Motivation is important to keep up, and this will effect us in the long run through the course of the project.
\cite[P.~95-97]{ProjectManagement}

\subsection{The Team's Goal and Ambitions}
\label{sec:team}
When it comes to the team, it is important not to use the terms \textit{``Us"} and \textit{``Them"}, because the team should be viewed as one. Don't have too high expectations for the project, and have clear goals and ambitions. The importance of not to over-scope can never be emphasized enough, and that everyone has agreed on it. Prioritize what is important and set up a deadline, and eventually extend it if necessary.\cite[P.~97-109]{ProjectManagement}

\subsection{The Roles and Responsibility}
\label{sec:rolesAndResponsibility}
Get to know your team and set the roles and give each team member their responsibilities  and decide the project leader. Remember that communication is the key, and that a team does not get great just like that, it needs work. The reason why some team fail is rarely caused by lack of knowledge, it is more often the lack of communications and wrong focus. These things can lead to disagreement and frustration. Trust in a team is important in a team, and whats holds it together. \cite[P.~144, P.~155-157]{ProjectManagement}

