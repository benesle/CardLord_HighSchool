\chapter{PreProduction}
\label{ch:kap7}

\section{The different phases of development}
\label{sec:DevelopmentPhases}
The beginning of a developing phase is never easy, especially when the team member doesn't know each other. A good place to start is getting to know your team, get a better understating of them and their weaknesses and strengths. When it comes to the composition of our team, it has a lot of participants which means an increased risk for bigger differences in personality traits. It gives a higher risk for conflicts, but increased potential for creativity. Teams in this size has an even bigger demand for structure and management, than smaller groups.\cite[P.~39]{ProjectManagement}

When it comes to getting started on the development of the game itself, there's a couple of things we need to get in order. 
First of all, if you decide to develop a game it should be for the right reasons. And with that in mind, we must ask ourselves \textbf{"Why are we making this game?"}.
What I mean by saying \textit{wrong} reasons to make games, I'm talking about bad ones that are almost certain to be unsuccessful and would just be a wast of time and money.\cite[P.~14]{GamificationFieldbook}

\subsection{Bad and Good Reasons}
\label{sec:GoodBad}
I think we can all agree that games are fun and that is what they are meant to be. The thing is just because something is entertaining or fun, doesn't mean that you will get something out of it. It takes a lot of time, it also might not ever be completed and will just end up in a pile of all your \textit{unfinished projects}. Although fun can be a part of the final solution, but should not be what drives it. The effort that it acquires to develop games, must be driven by internal motivation and not just because "everybody" is doing it. You may seems like there is a new release every time you turn. Everywhere you go you see articles, on TV or in press release of another college group or a small company with a new game implementation. That we really know isn't true, because then we would have seen a lot more of games than we do. Don't just dive right into it, without planning, motivation and knowledge. Because in the end it would just cost us a lot of resources and require a lot of our energy and attention. In other words creating a game would require a lot of us and our full attention. \cite[P~14-17]{GamificationFieldbook}.

Believe or not, everybody does not love or even like games. It may seem completely unthinkable, but the world consists of many different individuals, and you can't please them all. That doesn't mean that you should drop your project, because someone or some people didn't like it. The point here is that you can't create a game with the argument that \textit{everyone will love it}. It would be far more effective to make an argument related to the new opportunities, increase in performance  or something innovative. And if you think that creating a game is easy, you couldn't be more wrong. It is a long difficult and time consuming process, there is not only your focus and motivation you have too keep up, but the player too. A key thing to take from this is that  when a game is intuitive, are relative easy to play and understand, there is a long process behind it with a lot of complexities. Just remember that the process of developing a game is more complicated than you originally thought.\cite[P~18-20]{GamificationFieldbook}.

\section{Concept art}

\section{Programming}
\label{sec:PreProgrammin}
In the pre-production, there was one more programme, then there is now at the end. That complicated things and changes a lot of the work load. It was suddenly a lot more pressure, now that there were one of us less. Which meant that the assignments that was originally intended for three, needed to be divided among the two remaining programmers. 

\section{Team Work}
