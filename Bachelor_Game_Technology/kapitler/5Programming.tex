\chapter{Programming}
\label{ch:kap5}

\section{Game system}
\label{sec:Gamesystem}

\subsubsection{The Character}
The first thing that was implemented was our Character class, which we called "MyCharacter". This handles the character movements and  how it's working. 

The "MyCharacter" has mechanics for moving in each direction. The ability to move forward, backward, left and right. There is also a Camera following the Character which is attach to a Camera Boom, this will allow the camera to move closer character when needed, in case of blocking the players view. The Character also have a Collision capsule for colliding with collideable objects in the interactive Game World.  

\subsubsection{The Player Controller}
Then there is the Player Controller, which is responsible for translating the input from the player into the character. This will make the character perform the actions the player wants it to do.

\subsubsection{GameMode} 
This GameMode is the default GameMode, then we make a Blueprint instance of this class. This exposes the GameMode properties to the Blueprint, and makes it much easier to modify those properties. In the GameMode we overrides the "BeginPlay" and use it to call the "init" function. In the GameMode class we created a combat function, first we located the enemies DataTable and picks the one with the corresponding ID, it then constructs a new GameCharacter and then adds the new enemyCharacter to the list of enemies. It then creates a new instance of the Combat Manager, going through the player group and the enemy group.

\subsubsection{GameCharacter}
This Object class called "GameCharacter" is a way to get the Character's current state and to see stats such as Health. This class have access to the data in the DataTables we have, such the ones for the "Player", "Enemy" and "CharacterClass". We use macros as UCLASS and UPROPERTY to expose all of this information to the Blueprint. In our GameCharacter class we declare our "CreateGameCharacter" factory that has a pointer to the Character DataTable and then tries to find the Characters class data.

\subsubsection{GameInstance}
This is usefull for keeping track of the character's stats and it's group members. It's better to use a GameInstance rather than the GameMode, because it precist through the game, and not loose information between level loads. The GameInstance here is called "CardLordGameInstance" This will hold information such as GroupMembers and the inventory.

\subsubsection{Combat Manager}
This will help us with organizing combat and handle it for us. The Combat Manager gets created on encounter, the combat begins and then deleted after the combat is done. We also add a bool to determine whether or not the Character is a Player or an Enemy.

Then to keep track of all of this visually, we made the CombatUI, the Character and the Enemy has their own status panel to display stats such as Health and Stamina.

All of the things with UI were done with Blueprint to not make it too complicated. This means things like, buttons and different kinds of UI elements.


\subsubsection{DataTable}
We will first begin with defining the stats in the Character DataTable Class called "FCharacterData" and for the enemy "FEnemyData" and for the Character Classes "FCharacterClassData". We will make all of the values available in Blueprint. We are then able to make a DataTable and use these data classes as the DataTable structure, and can add the Character and Enemies we want.

\subsection{The Combat System}
It is the Combat Manager who handle the combat for us.
The Combat itself will be divided into two main GameSates called "Decision" and "Action". That is Decision state, where the characters decides their action. The Action state is where all the characters chosen action will be done. The Combat Manager has three TArrays, one of them is for the turn order, the other one keeps tracks of the players and the third one has with the GameState to do.

There are two main states and that's Decision and Action. Every round starts with the Decision making and switch state to Action where all their chosen actions is performed. The GameOver and Victory state on the other hand is for when either all the enemies are dead or when all the players are dead. That is the reason why there are a list for each of the groups.

The Combat Manager also has a tick function, and will be called by the Game Mode every frame until the Combat is over. The "currentTickTarget" and "tickTargetIndex, they will have a pointer on one character, like in the Decision state will this pointer start with the first Character in the turn order, and will return true when the character has made their decision. The pointer will then move to the next Character until all the Characters have decided on an action. The Combat Manager will then switch the state to Action state.

The "SelectNextTarget" function will start at the current tickTargetIndex, and find the first Character in the turn order that is not dead. If there is no one to be found the tickTargetIndex will be set to -1 and the tick to a nullPtr. 
\cite{BuildingAnRPG}

\section{process of improvement}
\label{sec:improvements}
There has been continuously attempts on improving the Combat, Inventory and Shop and how we save the game. Some of the "fixes" might have lead to another bug that weren't there before. Working with the Unreal Engine has been a challenge. There were incidences with bugs that did not make any sense, and which was later fixed by deleting all of the building files and generate new Visual studio files. When we in addition add GitHub in the mix, there ended up with a lot of "merge conflicts" and some overridden progress for some. This got better as we went further into the project work, and got more familiar with the program and how it works.


